\chapter{Appendix for Chapter 3}\label{AppA}
\myappendices{Appendix \ref{AppA} }%\byname{AppA}}
\section [Derivation of the PDE model and its steady state solution]{Derivation of the PDE \ref{pde} and its steady state solution }\label{a1}
Let $Q(x,t)$ be the length distribution of prophages of length $x,$ at a time $t.$ Then after time step, $\delta t,$ we have
\begin{eqnarray}\label{d1}
Q(x, t+\delta t) &=& Q(x,t)+ D(x+\delta x)\,P(x+\delta x, t)\, \frac{\delta t}{\delta x}-D(x)\,Q(x,t)\,\frac{\delta t}{\delta x}+ r_{S}S(x)\,Q(x,t)\,\delta t\nonumber\\
&-&r_{I}I(x)\,Q(x,t) \, \delta t+\alpha \, f(x)\, \delta t + \beta \, g(x)\, \delta t. \nonumber
\end{eqnarray}
Using Tylor's series expansion and after simplification we arrive at the following 
\begin{eqnarray}
\frac{Q(x,t+\delta t)- Q(x,t)}{\delta t}& = &   \left(\frac{D(x+\delta x)-D(x)}{\delta x}\right)\,Q(x,t)+D(x+\delta x)\,\frac{\partial Q(x,t)}{\partial x}+ \mathcal{O}(\delta x)\nonumber\\%&+&\mathcal{O}(\delta x)^2+...+r_{S}S(x)\,Q(x,t)-r_{I}I(x)\,Q(x,t)+\alpha \,f(x) \nonumber
\end{eqnarray}
Now taking $\lim_{\delta t\to 0} $ and $\lim_{\delta x\to 0},$ we arrive at
\begin{eqnarray}\label{apppde}
\frac{\partial Q(x,t)}{\partial t} &=& \frac{\partial D(x)}{\partial x}\,Q(x,t)+D(x)\,\frac{\partial Q(x,t)}{\partial x}+r_{S}S(x)\,Q(x,t)-r_{I}I(x)\,Q(x,t)+\alpha \, f(x) + \beta \, g(x)\nonumber\\ 
&=& \frac{\partial }{\partial x}[D(x)\, Q(x,t)]+ [r_{S}S(x)-r_{I}I(x)]\,Q(x,t)+\alpha \, f(x)+ \beta \, g(x).
\end{eqnarray}
If we consider $\lim_{t \to \infty}\,Q(x, t)=P(x)$ and  $D(x) = r_{D}\,x$
then the differential equation generating steady state solution, of the PDE \ref{pde}, is given by \begin{eqnarray}\label{appsol1}
 \frac{d P(x) }{d x}+ \left(\frac{1}{x}+\mathcal{F}(x)\right)\,P(x)+\frac{\alpha}{r_{D}\, x} \, f(x)  + \frac{\beta}{r_{D}\,x} \, g(x) =0
\end{eqnarray}
where $\mathcal{F}(x) = \frac{r_S\, S(x)}{r_D\,x }- \frac{r_{I}\, I(x)}{r_D\, x}.$
Equation (\ref{appsol}) is first order linear ODE and its solution is given by 
\begin{eqnarray}\label{ssol1}
P(x) &=&\frac{-\,e^{-\int{\mathcal{F}}(x)dx} }{r_D\, x}\,\int{(\alpha \,f(x)\, + \, \beta\, g(x))\,e^{\int{\mathcal{F}}(x)dx}dx}+\frac{C}{x}\, e^{-\int{\mathcal{F}}(x)dx},
\end{eqnarray}
where $C$ is a constant of integration.
\section{Results from model selection and data fitting}\label{a2}
The AIC value is the measure of loss of information for the model under consideration and is an ordinal number, used for ranking models. The lowest AIC value corresponds to the best fit.  If the number of data points are small enough compared to the number of parameters then the AIC value is not penalized enough. To remedy this problem a second order Akaike Information criteria, the corrected Akaike Information Criteria (AICc), is defined. The corrected  Akaike Information Criteria (AICc) is given as \citep{burnham_model_2003}:
\begin{eqnarray}\label{aicc}
AIC_c=AIC + \frac{2k(k+1)}{n-(k+1)}.
\end{eqnarray}
As the number of data points becomes large enough, AICc values converge to AIC values and either of these criteria can be used to determine the best fit model amongst the candidate models \citep{burnham_model_2003}. In the tables to follow, we provide both AIC and AICc values, and compute relative probabilities using the AICc values.
  \subsection{Data Set 1}
  \begin{table}[hbt!]
\centering
\begin{tabular}{ p{1cm}p{2cm}p{2cm}p{2cm}p{2cm}p{3cm}  }
\hline
\# & Parameters & AIC & AICc& Log-likelihood & Relative probability (AICc) \\
\hline
1&                        15&          4884.1734&          4884.9642&         -2427.0867&         0.3719\\
2&                        14&          4882.2952&          4882.9860&         -2427.1476&         1\\
3&                        12&          4893.5207&          4894.0322&         -2434.7603&       0.0039\\
4&                        11&          4894.7613&          4895.1920&         -2436.3807&       0.0022\\
5&                         9&          4908.5087&          4908.8014&         -2445.2544&      2.4788e-06\\
6&                         8&          4906.2759&          4906.5097&         -2445.1379&      7.7964e-06\\
7&                         6&          5044.8136&          5044.9499&         -2515.9068&      6.7604e-36\\
8&                         6&          5069.6683&          5069.8042&         -2528.8341&      2.7098e-41\\
9&                         4&          5063.9143&          5063.9788&         -2527.9571&      4.9878e-40\\
\hline
\end{tabular}
\caption[Number of parameters, AIC, AICc values, log-liklihood and the corresponding relative probabilities for Data Set 1.]{Number of parameters, AIC, AICc values, log-liklihood and the corresponding relative probabilities for Data Set 1 \citep{bobay_pervasive_2014}. The best fit model includes a mixed distribution to describe autonomous temperate phages ($g$=3), degradation, induction and selection. The second best fit model is the same model with HGT and has relative probability $0.3791$.}
\label{table:bob}
\end{table}
\newpage
\subsection{Data Set 2}
\begin{table}[hbt!]
\centering
\begin{tabular}{ p{1cm}p{2cm}p{2cm}p{2cm}p{2cm}p{2cm}  }
\hline
\# & Number of Parameters & AIC & AICc& Log-likelihood  & Relative probability \\
\hline
1&                         15&          993.726&          998.583&         -480.863&       0.0016\\
2&                        14&          990.549&          994.797&         -480.275&        0.0108\\
3&                        12&          991.327&          994.492&         -482.663&         0.0126\\
4&                        11&           987.247&          989.937&         -481.624&         0.123\\
5&                         9&          987.197&          989.062&          -483.599&         0.191\\
6&                         8&          984.237&          985.749&         -483.119&          1\\
7&                         6&          1006.929&          1007.855&         -496.464&      1.585e-05\\
8&                         6&          1012.234&          1013.159&         -499.117&       1.117e-06\\
9&                         4&          1004.729&          1005.216&         -497.364&      5.927e-05\\
\hline
\end{tabular}
\caption[Number of parameters, AIC, AICc values and the corresponding relative probabilities for Data Set 2.]{Number of parameters, AIC, AICc values and the corresponding relative probabilities for Data Set 2 \citep{crispim_screening_2018}. The best fit model includes degradation, induction and selection as well as one Gaussian distribution to describe autonomous temperate phages ($g$=1).}
\label{table:desu}
\end{table}
\newpage
\subsection{Data Set 3}
\begin{table}[hbt!]
\centering
\begin{tabular}{ p{1cm}p{2cm}p{2cm}p{2cm}p{2cm}p{2cm}  }
\hline
\# & Number of Parameters & AIC & AICc& Log-likelihood  & Relative probability \\
\hline

 1&                       15&          5671.819&          5672.579&         -2819.909&        0.0318\\
 2&                       14&          5669.819&          5670.489&         -2819.909&        0.0904\\
 3&                       12&          5670.179&          5670.685&         -2822.089&        0.0819\\
 4&                       11&          5667.438&          5667.872&         -2821.719&          0.3347\\
 5&                        9&          5667.461&          5667.766&         -2823.731&         0.3528\\
 6&                        8&          5665.434&          5665.683&         -2823.717&         1\\
 7&                        6&          5731.084&          5731.239&         -2858.542&      5.8167e-15\\
 8&                        6&          5757.726&          5757.880&         -2871.863&      9.5404e-21\\
 9&                        4&          5753.680&          5753.762&         -2871.840&      7.4776e-20\\
\hline
\end{tabular}
\caption[Number of parameters, AIC, AICc values and the corresponding relative probabilities for Data Set 3.]{Number of parameters, AIC, AICc values and the corresponding relative probabilities for Data Set 3 \citep{leplae_aclame:_2010}. The best fit model includes degradation, induction and selection as well as one Gaussian distribution to describe autonomous temperate phages ($g$=1). }
\label{table:aclame}
\end{table}
\addcontentsline{toc}{chapter}{Bibliography}
\bibliographystyle{apa}
\bibliography{refrence}

