\chapter{Discussion \& Conclusions}
The study of phage-bacteria interactions involves gigantic numbers \cite{noauthor_microbiology_2011}: if all the viruses on earth were laid end to end they would stretch for 100 million light years; there are $13\times10^{28}$ bacteria in the oceans (100 million times more than the number of stars in the known universe); $1\times10^{31}$ viral infections occur per second in the ocean environment which results in the removal of  20\% to 40\% of the bacterial mass in the oceans per day;  one gram dental plaque contains approximately $1\times10^{11}$ bacteria; a teaspoon of soil  contains $1\times 10^{9}$ microorganisms; bacteria present in the human gut weigh about 1 kg; 8\% of human DNA is of viral origin;  we have only sequenced $1\times10^{-22}$\% of the total DNA on earth.

These quantities, along with the importance of the microorganisms to our ecosystem \cite{gibbons_microbial_2015, graham_microbes_2016}, public health \cite{smith_microbiology_2015}, food, and the possible discovery of new life on distant planets \cite{lopez_inevitable_2019}, have transformed our views about phage-bacteria interaction from being a simple system to a complex and important set of interactions. The intriguing dynamics of phage-bacteria interactions and the urgency of these global issues have invited us to dive deep into this ocean of knowledge. Once immersed in this study, we realized that there are more surprising facts, such as: the amount of temperate phage DNA in bacterial genomes surpasses the amount of DNA in free-living phage \cite{wahl_prophage_2017}; and pathogenic bacteria, responsible for the death and miseries of millions of people,  are domesticating these viral genomes with high frequency \cite{fortier_importance_2013}. In this thesis, we contributed to this vast field. Our contribution is in the form of three projects, called \textbf{Chapter 2}, \textbf{Chapter 3} and \textbf{Chapter 4}, in this investigation. Below we provide details of these projects and the conclusions derived from them. 

Here, we started with a topic directly related to public health, “Phage therapy and antibiotics for biofilm eradication: a predictive model” (\textbf{Chapter 2}). In that chapter, we developed a simple predictive model to capture the effect of the synergistic use of phages and antibiotics in biofilms. For this study, we assumed that bacteria are offering structural resistance (by grouping together, constructing biofilm and developing an EPS structure around the biofilm) to the antibiotic. In this model, we also used the idea of a group defense mechanism, that is, a phage functional response to bacteria resembling a Holling type IV functional response. We were able to show that neither antibiotic nor phages alone can eliminate the biofilm completely, and the synergistic effect of applying phages first and then antibiotics works better. 

Prophages, being the main source of bacterial genome diversity and important factors in bacterial genome evolution, are more abundant in bacterial genomes than previously thought \cite{costa_genomic_2018, mottawea_salmonella_2018}. In \textbf{Chapter 3}, we developed a mathematical model to study the effect of various evolutionary forces acting on prophages. We investigated the model in detail, fitting against some publicly available datasets, and were able to quantify the relative rates of these evolutionary forces in time units expressed in terms of the ``expected prophage lifetime", that is, the average time between lysogeny and induction (see Table \ref{tab:rates}). From these rates, we conclude that:   (1) the time between lysogeny events is about 5 prophage lifetimes, that is, new prophages enter into bacterial genomes at a rate one-fifth of the induction rate; (2) the selection coefficient is 0.5 per prophage lifetime;  (3) on average a prophage has lost only 1\% of its genome at the time of induction; (4) a minimum of two to three prophage genes are required to excise prophage from the bacterial genome.  The relation between prophages and their bacterial hosts is defined to be parasitic or mutualistic depending on the balance between the cost and benefits of the integration of foreign DNA \cite{shapiro_evolution_2018}. The biggest cost due to prophage integration is the possibility of induction, which results in the killing of the bacterial host, although there are other small costs as well, such as energy costs \cite{koonin_evolution_2009}. Our results predict a tipping point between parasitism and mutualism, the point at which cost equals benefit. Our model predicted that the bimodal prophage size distribution is due to the balance between selective advantage (benefit) and induction (cost). The peak on right is due to the lysogeny of new prophages and the peak on left is due to the accumulation of smaller prophages, conferring more benefits to the bacterial hosts than their cost, as shown in Figure \ref{fig:combine}. 

The domestication of defective prophages, the retention of defective prophages that confer some benefit to their hosts, is a common phenomenon in bacterial populations \cite{bobay_pervasive_2014}. We believe that these defective prophages have a prominent role in shaping bacterial genome evolution. The genetic material of domesticated defective prophages may also serve as a tool‐box for other prophages to use for repair through recombination \cite{de_paepe_temperate_2014}. In \textbf{Chapter 4}, ``The genetic repertoire of prophages'' we focus on genes enriched in smaller prophages and the role of evolutionary forces. First, we downloaded data regarding the genetic repertoire for two well-studied prophage databases \cite{bobay_pervasive_2014, leplae_aclame:_2010}, using PHASTER \cite{arndt_phaster:_2016}. The distributions of 53,356 annotated prophage genes identified in 1384 prophage sequences were examined, showing that: (1) genes involved in the lytic life cycle were preferentially lost in smaller prophages; (2) transposes and integrases are significantly enriched in smaller prophages. We also developed an ODE model and computational model to study the effect of these evolutionary forces on the genetic repertoire of prophages.  While the models were able to explain many interesting features of the data, we were unable to explain the enrichment of integrase genes in smaller prophages. 

\section{Future Work}

We believe that the ODE model \ref{modelb}, representing phage-bacteria interaction in bacterial biofilm colonies may have some rich dynamics. We are planning to extend this work further and carry out further detailed bifurcation analysis of the system. In our model, phages and bacteria interact with each other according to the law of mass action and we ignore spatial structure. Several studies have concluded that ignoring the spatial structure of a biological system may lead to inaccuracies \cite{dieckmann_geometry_2000}, and bacterial biofilms have a complex and interesting structure \cite{flemming_biofilms:_2016}. The inclusion of spatial structure in this model is needed to better understand the dynamics of interaction between phages and bacteria in bacterial biofilm colonies.

Prophages are abundant in bacterial genomes and are particularly prominent in pathogenic bacterial genomes \cite{canchaya_impact_2004, brussow_phages_2004}. Prophages from pathogenic bacteria have been shown to encode virulence factors \cite{hyman_bacteriophages_2012}. The insertion of these extra genes in bacterial genomes may increase the bacterial genome size.  But studies have shown that pathogenic bacteria have smaller genomes and fewer genes than their closest non-pathogenic relatives \cite{ moran_microbial_2002, toft_evolutionary_2010}. Gene acquisition and deletion may be the events underlying the emergence and evolution of bacterial pathogens \cite{ochman_genes_2001}. The fact that despite having more prophages in their genomes, pathogen genomes are smaller than the closest non-pathogen relative gives rise to a question: what is the relation between prophages and rates of gene gain and loss, acting on the whole host genome?. The availability of a huge amount of data regarding bacterial pathogens and the usefulness of mathematical modeling can give insights into this question. 

Similarly, mutational deletion is a prominent evolutionary force acting on prophages.  Such mutations make these prophages shorter, eventually resulting in domestication or deletion from the bacterial genome \cite{bobay_pervasive_2014}. Prophages can excise from the bacterial genome randomly or due to some DNA damaging agent, resulting in free life as a temperate phage \cite{nanda_impact_2015}. Does this mutational deletion cause a reduction, over evolutionary time, in the genome size of temperate phages? If not, how can temperate phage keep their genomes intact in the presence of mutational deletions as an important evolutionary force acting on prophages? In other words, what is the role of mutational deletions in the evolution of temperate phages? 

One of the strongest signals obtained from the genetic repertoire data of prophages, in \textbf{Chapter 4}, was the significant enrichment of integrase genes in incomplete prophages (see Table \ref{tab:genes} and Figure~\ref{fig:data}). Full and partial prophage sequences are frequently transferred horizontally through transduction \cite{ fillol-salom_bacteriophages_2019} and related processes such as molecular parasitism by GTAs \cite{lang_gene_2012}. Once foreign DNA enters a host cell through HGT, it needs to recombine with the host genome; integrase acts as a catalyst to mediate the process of recombination, resulting in an increased rate of recombination. Therefore, integrases likely help prophage-derived elements with recombination into the host genome during horizontal transmission. The horizontal transmission of integrase genes along with their neighboring prophage genes may have caused significant enrichment of integrase in incomplete prophages. From this, we hypothesize that in the event of horizontal transmission, a mobile genetic element containing the integrase gene may have a selective advantage over a mobile genetic element lacking the integrase gene. We are planning to explore this hypothesis in much more detail, through an individual-based model like the one developed in Chapter 4, but including more detail such as horizontal gene transfer.

\addcontentsline{toc}{chapter}{Bibliography}
\bibliographystyle{abbrv}
\bibliography{refrence}